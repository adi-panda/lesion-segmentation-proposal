\par
Stroke is currently the second leading cause of death worldwide. 
Early localizaiton of lesion plays a vital role in prediction of
the final infarct volume and the clinical outcome of the stroke
patient. As demonsrated in \cite{wheeler_early_2013}, growth of the lesion
volume during the initial MRI screenings, demonstrates a strong coorelation
with the final infarct volume. However, prediction of the final infarct volume
can be challening due to numerous factors including timing and the extent
of reperfusion. Additionally, determining volume in FLAIR images can be particularly
challenging due to the signficantly lower sensitivity to ischemic lesions in comparison
with DWI images \cite{kamalian_stroke_2019}\cite{noguchi_mri_1997}. Nonetheless, the ability to
accurately predict the final infarct volume can result in life saving decisions for the patient.
\par
Current techniques involving thresholding and Voxel-Based Morphometry \cite{noauthor_voxel-based_nodate}
for segmentation of stroke lesions can prove to be time consuming,
requiring manual input from a trained professional. Deep learning techniques were first
applied to stroke lesion segmentation in 2016 using a dual CNN model architecture
with results comparable to manual techniques. Since then, there have been a number of attempts
to improve upon the original model architecture, and with the release of the nnU-net\cite{isensee_nnu-net_2021}
architecture, there has been a significant improvement in the accuracy of these models. 



